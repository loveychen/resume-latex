%----------------------------------------------------------------------------------------
%	SECTION TITLE
%----------------------------------------------------------------------------------------

\cvsection{工作及项目经历}

%----------------------------------------------------------------------------------------
%	SECTION CONTENT
%----------------------------------------------------------------------------------------

\begin{cventries}
	
	%------------------------------------------------
	
	\cventry{算法专家} % Job title
	{360数科-数据AI中台-机器人组} % Organization
	{北京} % Location
	{2020-05 \textasciitilde 至今}
	{ % Description(s) of tasks/responsibilities
		\begin{cvitems}
			\item \textbf{机器人新架构}. 负责新架构设计及核心功能开发, 使用 gRPC 设计通信接口, 使用 Python / Golang / Java 实现具体功能
			\item \textbf{会话分析模型}. 负责会话分析模型 及 会话分析模型管理平台 的设计和开发, 模型 AUC 0.86+
			\item \textbf{响铃语音分类}. 使用 MFCC + VGGish 搭建响铃语音分类模型, 模型准召率 99+\%
			\item \textbf{获奖情况}. \textbf{2020 Q2 最佳新秀} 及 \textbf{2020 年度 A+ 绩效}
		\end{cvitems}
	}
	
	%------------------------------------------------    
	
	\cventry{高级算法工程师} % Job title
	{五八集团-TEG-AI Lab} % Organization
	{北京} % Location
	{2018-04 \textasciitilde 2019-10} % Date(s)
	{ % Description(s) of tasks/responsibilities
		\begin{cvitems}
			\item{\textbf{语音机器人项目负责人}. 带领项目组完成项目从零到一的创造性突破. }
			\item{\textbf{负责语音机器人系统架构设计}. 划分了 通信层/对话层/接入层/基础技术层 四个模块, 模块间使用事件机制通信}
			\item{\textbf{对话管理系统设计与开发}. 参考业界聊天机器人通用架构, 主要创新点包括: 1. 整轮会话质量评估, 包括整轮意图 / 对话完成度 和 业务标签, 方便人工有选择跟进;  2. 使用 DAG 配置话术流程, 非技术人员可以通过 Web 配置方式定制不同的机器人}
			\item{\textbf{整轮意图识别模型设计与开发}. 主要负责黄页销售业务线的整轮意图识别, 先后迭代了 TextCNN / Wide\&Deep / BERT 等模型}
			\item{\textbf{团队建设}. 带动组员完成多次内部分享, 撰写并分享了多篇对外宣传技术文章, 注重知识积累, 让每个同学都能在项目中学到知识, 发挥自己的闪光点(示例文章: \postlink{http://dwz.date/GM5}{智能语音机器人中对话管理系统设计与实现}, \postlink{http://dwz.date/GMC}{智能语音机器人架构实践}) }
			\item{\textbf{获奖情况}. 获得 \textbf{2018 年度突出贡献员工} 和 \textbf{2018 年度优秀团队}}
		\end{cvitems}
	}
	
	%------------------------------------------------
	
	\cventry{高级研发工程师} % Job title
	{融360-基础技术部} % Organization
	{北京} % Location
	{2016-06 \textasciitilde 2018-04} % Date(s)
	{ % Description(s) of tasks/responsibilities
		\begin{cvitems}
			\item {负责三大运营商用户数据的\textbf{授权爬虫}, 为企业风控模型提供数据支持}
			\item {设计并开发\textbf{图像验证码自动识别模型}, 平均识别准确率 97\% 以上}
			\item {设计并开发\textbf{爬虫信息完整性模型}, 节省计算资源 60\% 以上}
		\end{cvitems} 
	}
	
	%------------------------------------------------
	
\end{cventries}
